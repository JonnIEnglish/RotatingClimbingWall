\section*{Executive summary}
\noindent
\rowcolors{2}{blue!5}{white}
\begin{longtable}{|p{\dimexpr \linewidth-2\tabcolsep-2\arrayrulewidth}|}
\hline%------------------------------------------------------------
\sumheading  Title of Project \\
\hline%------------------------------------------------------------
Development of a Rotary Climbing Wall \\[1ex]

\hline%------------------------------------------------------------
\sumheading  Objectives \\
\hline%------------------------------------------------------------
To design and prototype a rotary climbing wall to enable continuous and long-duration climbing, facilitating effective endurance training indoors.\\[1ex]

\hline%------------------------------------------------------------
\sumheading  What is current practice and what are its limitations? \\
\hline%------------------------------------------------------------
Current indoor climbing practices involve static walls with limited routes, which restrict continuous climbing and endurance training due to spatial constraints.  \\[1ex]

\hline%------------------------------------------------------------
\sumheading  What is new in this project? \\
\hline%------------------------------------------------------------
The project introduces a rotating wall mechanism that would function similarly to a vertical treadmill, allowing for uninterrupted climbing and customizable difficulty levels. \\[1ex]

\hline%------------------------------------------------------------
\sumheading  If the project is successful, how will it make a difference? \\
\hline%------------------------------------------------------------
The successful implementation of this project will enable training devices such as this to be more accessible at a lower price.\\[1ex]

\hline%------------------------------------------------------------
\sumheading  What are the risks to the project being a success? Why is it expected to be successful? \\
\hline%------------------------------------------------------------
Risks involve the project becoming too expensive and being unable to be completed or viably built in the time available or within the allocated budged. 
\\[1ex]

\hline%------------------------------------------------------------
\sumheading  What contributions have/will other students made/make? \\
\hline%------------------------------------------------------------
This research project will serve to contribute to Stellenbosch University's knowledge and experience when it comes to rock climbing training techniques and equipment.
 \\[1ex]

\hline%------------------------------------------------------------
\sumheading  Which aspects of the project will carry on after completion and why? \\
\hline%------------------------------------------------------------
Further research may continue in optimizing the wall's design, exploring commercial production possibilities.\\[1ex]

\hline%------------------------------------------------------------
\sumheading  What arrangements have been/will be made to expedite continuation? \\
\hline%------------------------------------------------------------
 Documentation of design and development processes will be created in enough detail that the production demands and marketing potential can be assessed. \\[1ex]

\hline%------------------------------------------------------------
\end{longtable}


\newpage
\section*{ECSA Outcome Self-Assessment}

\begin{table}[H]
\centering
\label{tab:ecsa-outcome-self-assessment}
\resizebox{\textwidth}{!}{%
\begin{tabular}{|>{\raggedright\arraybackslash}p{0.2\linewidth}|p{0.5\linewidth}|>{\raggedright\arraybackslash}p{0.25\linewidth}|}
\hline
\textbf{ECSA Outcome} & \textbf{Description} & \textbf{Addressed in Sections} \\ \hline
ELO 1: Problem Solving & Demonstrate competence to identify, assess, formulate, and solve convergent and divergent engineering problems creatively and innovatively. & 4, 5 \\ \hline
ELO 2: Application of Scientific and Engineering Knowledge & Demonstrate competence to apply knowledge of mathematics, basic science, and engineering sciences from first principles to solve engineering problems. & 4, 5, Appendices: A, B, C, D \\ \hline
ELO 3: Engineering Design & Demonstrate competence to perform creative, procedural, and non-procedural design and synthesis of components, systems, engineering works, products, or processes. & 5 \\ \hline
ELO 5: Engineering Methods, Skills, and Tools & Demonstrate competence to use appropriate engineering methods, skills, and tools, including those based on information technology. & 3, 4, 5, 6 \\ \hline
ELO 6: Professional and Technical Communication & Demonstrate competence to communicate effectively, both orally and in writing, with engineering audiences and the community at large. & Project Proposal, Progress Report, Progress Presentation, Final Report, Final Presentation, Video, Weekly meetings \\ \hline
ELO 8: Individual, Team, and Multidisciplinary Working & Demonstrate competence to work effectively as an individual, in teams, and in multidisciplinary environments. & All elements of the project \\ \hline
ELO 9: Independent Learning Ability & Demonstrate competence to engage in independent learning through well-developed learning skills. & 2, 3, 4, 7, Appendices: A, B, C, D, E, F \\ \hline
\end{tabular}%
}
\end{table}
