\chapter{Techno-economic Analysis}
\label{chap:techno_economic_analysis}

This chapter assesses the economic potential of introducing the rotating climbing wall to the South African market, with a brief look at international prospects.

\section{Market Overview}

\subsection{Global and South African Markets}

Rotating climbing walls are a niche within the fitness equipment industry, offering continuous climbing without tall structures. Key global competitors include Treadwall, XClimb Pro, and ClimbStation, which offer models with varying sizes, features, and pricing. In South Africa, the market is largely untapped, with a growing number of climbing gyms but limited access to advanced equipment like rotating walls.

\section{Product Comparison}

The proposed rotating wall features automatic incline and speed adjustments, a customizable control panel, and comprehensive safety features. Its compact design makes it cost-effective to scale, unlike larger and more expensive competitors.

\subsection{Comparison Table}

\begin{table}[H]
\centering
\caption{Comparison of Rotating Climbing Walls}
\label{tab:climbing_wall_comparison_economic}
\resizebox{\textwidth}{!}{%
\begin{tabular}{|p{0.15\linewidth}|c|c|c|c|c|p{0.15\linewidth}|c|c|}
\hline
\textbf{System} & \textbf{Width (m)} & \textbf{Height (m)} & \textbf{Incline Adjustment} & \textbf{Speed Control} & \textbf{Monitoring} & \textbf{Extra Functions} & \textbf{Material} & \textbf{Price (R)} \\
\hline
\textbf{Proposed Product} & 1.1 & 2.53 & Automatic & Automatic & Yes & Advanced Control Panel & Steel/Wood & \textbf{100,000} \\
\hline
XClimb Pro S & 1.5 & 2.82 & No & Yes & No & No & Steel/Wood & 230,000 \\
\hline
XClimb Pro XL & 2.0 & 3.48 & Automatic & Yes & No & No & Steel/Wood & 278,000 \\
\hline
ClimbStation & 1.9 & 3.3 & Automatic & Automatic & Yes & Advanced Control Panel & Aluminium & N/A \\
\hline
Treadwall V4 & 1.22 & 2.97 & No & Manual & Yes & Auto-stop & Steel/Wood & 202,000 \\
\hline
Treadwall Kore4/11 & 1.22 & 2.75 & Manual & Manual & Yes & Auto-stop & Steel/Wood & 140,600 \\
\hline
\end{tabular}
}
\end{table}

\section{Cost and Pricing Strategy}

The prototype costs approximately R52,400, with competitors priced between R140,600 and R278,000. The proposed price is R100,000, which undercuts competitors by nearly 50\%.

\textbf{Production Costs:}
\begin{itemize}
    \item \textbf{Materials}: R52,400
    \item \textbf{Labor}: R15,000
    \item \textbf{Overheads}: R5,000
\end{itemize}

Total production cost is R72,400. With a selling price of R100,000, the profit per unit is R27,600. The breakeven point is 8 units.

\section{Competitive Advantages}

The proposed rotating wall's key advantages are:
\begin{itemize}
    \item Full automation for incline and speed adjustments.
    \item Affordability, making it accessible to a wider market.
    \item Advanced safety features and customization options.
\end{itemize}

\section{Economic Feasibility}

Assuming a target market of 20 potential customers, capturing 50\% of the market within the first year (10 units) is achievable. The profit per unit sold provides a positive return on investment after the breakeven point.

\section{Challenges and Mitigation}

Key challenges include scaling production and creating market awareness. These can be mitigated by partnering with local manufacturers and running targeted marketing campaigns.

\section{Conclusion}

The analysis suggests that the rotating climbing wall is a viable and profitable product in the South African market. Its affordability, automation, and safety features provide a competitive edge, and initial investment can be recovered quickly with minimal sales.
