\chapter{Recommendations}

To enhance safety, it is recommended to implement additional protective mechanisms, particularly near the top where the panels rotate over the sprocket and converge. There is a potential risk of fingers getting pinched in this area; therefore, enclosing the top with a protective cover is advisable to prevent such incidents.

Incorporating optical sensors at the top and bottom of the wall is also suggested. These sensors would trigger when the climber reaches them, allowing the wall to automatically stop rotating when the bottom sensor is activated. This feature would enable climbers to rest or plan their next moves. Additionally, the sensors could be utilized for automatic speed control: increasing the rotation speed by a preset increment if the top sensor is triggered—indicating that the climber is ascending faster than the wall's rotation—and decreasing it if the bottom sensor is triggered.

Another recommendation is to develop a larger, more user-friendly control panel. Such an interface would allow users to preset training routes or routines that automatically measure the distance climbed and adjust the wall's inclination at different stages to simulate varying difficulties.

Implementing a smaller motor, after conducting the necessary tests to determine the required torque and power, could result in potential cost savings.

Furthermore, instead of dissipating all the generated energy through a braking resistor, the wall could harness this energy to charge a battery that powers all onboard electronics. This approach would eliminate the need for the wall to be connected to an external power source, making it entirely user-powered.

Finally, increasing the climbing area is recommended to enhance the user experience and accommodate a wider range of climbers.

\chapter{Conclusions}

This report addressed the complex engineering challenge of designing, manufacturing, and testing a full-scale, fully functional rotating climbing wall that rivals major competitors in both functionality and cost. The process involved investigating the problem, developing design constraints, generating a detailed design, manufacturing the system, and verifying its use and safety through testing.

The end product performed exceptionally well and exceeded expectations, successfully executing each intended function. A techno-economic analysis of launching the product to market was included, demonstrating its viability and competitive advantage. Additionally, recommendations were provided to further enhance the product's safety, functionality, and user experience.
