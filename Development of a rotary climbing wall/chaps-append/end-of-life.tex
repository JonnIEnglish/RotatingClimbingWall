\chapter{End-of-Life Strategy}

In the development of the rotary climbing wall, a key aspect of the planning and execution process is centred around sustainability and responsible use of resources. Recognizing that the environmental impact of the project is of importance, we will ensure that the process is carried out in a sustainable, environmentally friendly manner. This will be carried out all while not impacting the performance or safety of the end result.

\section*{Assembly Re-Use and Recycling}
The project will identify key areas that can take advantage of re-use and recycled/recyclable materials. One example of an area that could take advantage of re-use is approaching local climbing gyms and borrowing/buying climbing holds and plywood (a typical construction material at climbing gyms) from the gym as apposed to purchasing these items brand new. 

\section*{Disassembly Re-Use and Recycling}
The prototype will need to be disassembled or removed from the University at the end of the project. If it is disassembled, the climbing holds and plywood backing can be donated to or bought back by a local climbing gym. The power and control systems can be added back into the Mechanical/Mechatronic Department's inventory for future use. The same process could be applied to the frame.

