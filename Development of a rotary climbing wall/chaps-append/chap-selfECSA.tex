\chapter{Self ECSA Assessment}

{\small
\rowcolors{2}{black!15}{white}
\noindent
\begin{longtable}{|p{\dimexpr \linewidth-2\tabcolsep-2\arrayrulewidth}|}
\hline
\textbf{1. Problem solving}%: Demonstrate competence to identify, assess, formulate and solve convergent and divergent engineering problems creatively and innovatively. 
\\
\hline
 In this project I will identify the unique problems that come with designing and building a prototype rotary climbing wall. I will have to be creative and innovative with how I go about the process.
 \\[1ex]
\hline
\textbf{2. Application of scientific and engineering knowledge}%: Demonstrate competence to apply knowledge of mathematics, basic science and engineering sciences from first principles to solve engineering problems.
\\
\hline
I will be applying the engineering knowledge and practice in using first principles to solve engineering problems that I learned in third and fourth year to aid me in this project. Specifically from Machine Design, Applied Mathematics and Strength of Materials.
\\[1ex]

\hline
\textbf{3. Engineering Design}%: Demonstrate competence to perform creative, procedural and non-procedural design and synthesis of components, systems, engineering works, products or processes.
\\
\hline
The engineering design process will be followed when developing the rotating climbing wall. The problem will be broken down into sub-systems, each of which will be solved with procedural and non-procedural techniques.
\\[1ex]

\hline
\textbf{4. Engineering methods, skills and tools, including Information Technology}%: Demonstrate competence to use appropriate engineering methods, skills and tools, including those based on information technology.
\\
\hline
The project requires that I build a prototype of the rotating climbing wall. CAD design tool will be utilized in the design phase and numerous construction tools and methods will be used when creating the prototype.
\\[1ex]

\hline
\textbf{5. Professional and technical communication}%: Demonstrate competence to communicate effectively, both orally and in writing, with engineering audiences and the community at large.
\\
\hline

Professional and technical communication will be done in the reports one through four which will be written and produced by myself. I will be communicating in person with my supervisor in bi-weekly meetings about my progress and questions that I may have as well as the workshop staff for discussing getting parts made.
\\[1ex]


\hline
\textbf{6. Individual, team and multi-disciplinary working}%: Demonstrate competence to work effectively as an individual, in teams and in multi-disciplinary environments.
\\
\hline
Individual efforts will be  conducted through independent research and design concepts for the rotary wall. Collaboration will be done with workshop staff for the fabrication of parts and design advice, as well as with an academic supervisor to help with document writing. Multi-disciplinary actions may include consulting with actual rock climbers and athletes to discuss their training needs, in terms of a rotating wall.
\\[1ex]

\hline
\textbf{7. Independent learning ability}%: Demonstrate competence to engage in independent learning through well-developed learning skills.
\\
\hline
Engaging in self directed study to identify personal gaps in knowledge in terms of current rock climbing training techniques and relevant physiology. New software tools will be sought out and learned in order to create a high quality CAD model prior to construction. Research into motor and driver systems and  control system options will have to be completed.
\\[1ex]

\hline
\end{longtable}

}