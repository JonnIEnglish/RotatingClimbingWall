\chapter{Project Risk Assessment}

This section identifies and outlines potential risks that may negatively impact the project and its ability to be completed.

\begin{itemize}    
    \item \textbf{System Complexity:} The system is quite complex with many moving part that all need to work simultaneously. This creates a risk that if a single part of the system fails or is unable to be completed, the whole system may never work. This can be mitigated via a rigours design and testing process.
    
    \item \textbf{Project Timeline Overrun:} Again, due to the complexity of the project it is possible that the amount of work that needs to be done in order to build a functioning, full-sized prototype will be greater that the time available. If it is clear that this is the case, a scale model could be built as a prototype instead.
    
    \item \textbf{Budget Overrun:} Unexpected costs such as components being more expensive than anticipated, or duplicates of components needing to be bought to replace faulty is a possible risk. Mitigation may include applying for additional budget or self-funding the overshoot.
    
    \item \textbf{User Safety:} Due to the system being quite large with motor that has the potential to provide enough torque to lift a fully grown human. There is risk that the system may perform erratically, injuring the people around it. A safety system and a power cutoff switch must be implemented into the design.
    

\end{itemize}



